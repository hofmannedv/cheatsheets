% Essential Linux Commands
% Version 0.16
% (C) 2023-2024 Frank Hofmann <info@efho.de>
% Available from https://github.com/hofmannedv/cheatsheets

% Published under Creative Commons CC-BY-SA 4.0 International License
% https://creativecommons.org/licenses/by-sa/4.0/

\documentclass[10pt,a4paper]{article}
\usepackage{cheatsheeta4}

\renewcommand{\cheatsheetTitle}{Essentielle Kommandos \newline für Linux}
\renewcommand{\cheatsheetVersion}{Version 0.16}
\renewcommand{\cheatsheetCopyright}{
  \copyright~ 2023-2024 Frank Hofmann \faIcon{envelope} \href{mailto:info@efho.de}{info@efho.de} \\ Veröffentlicht unter 
 \href{https://creativecommons.org/licenses/by-sa/4.0/}{Creative Commons CC-BY-SA 4.0 \\ International License}. Erstellt mit \LaTeX. \\ \faIcon{github} \href{https://github.com/hofmannedv/cheatsheets}{https://github.com/hofmannedv/cheatsheets}~.
}

% add document metadata as XMP data
% add keywords based on hyperref package
\usepackage{hyperxmp}
\hypersetup{
    pdfauthor={Frank Hofmann}, 
    pdftitle={Essentielle Kommandos für Linux},
    pdfkeywords={Linux, Kommandozeile, Umgang mit der Shell, Paketverwaltung},
    pdfsubject={Effektiver Umgang mit der Linux-Kommandozeile},
    pdfcopyright={Copyright (C) 2023-2024 Frank Hofmann. Veröffentlicht unter Creative Commons Attribution - Share Alike 4.0 International License (CC-BY-SA-4.0)}
    pdfcopyrighturl={https://creativecommons.org/licenses/by-sa/4.0/legalcode.en}
}

\begin{document}

\cheatsheet

\begin{multicols}{2}

\section{Systeminformationen und -administration}
\begin{tabular}{ p{2.5cm} p{8.5cm} }
  \hline
  \texttt{blkid} & Gib die Blockgeräte samt deren IDs aus\\
  \rowcolor{Gray}
  \texttt{cfdisk}, \texttt{fdisk} & Ändere die Partitionstabelle \\
  \texttt{cpu-info}, \texttt{cpuid}, \texttt{lscpu} & Gib detaillierte Informationen zu den Prozessoren aus \\
  \rowcolor{Gray}
  \texttt{df}, \texttt{dfc}, \texttt{duf} & Zeige den freien Speicherplatz von Datenträgern an \\
  \texttt{du}, \texttt{duf} & Zeige den belegten Speicherplatz von Datenträgern an \\
  \rowcolor{Gray}
  \texttt{free} & Zeige sowohl den verfügbaren, als auch den benutzten RAM und Auslagerungsspeicher (Swap) an\\
  \texttt{halt}, \texttt{poweroff}, \texttt{shutdown} & Halte das System an\\
  \rowcolor{Gray}
  \texttt{journalctl} & Liefere Informationen zu den Logeinträgen via Systemd \\
  \texttt{lsblk} & Zeige Informationen zu den Blockgeräten an\\
  \rowcolor{Gray}
  \texttt{lsdev}, \texttt{lshw} & Zeige Informationen zu den installierten Hardwarekomponenten an\\
  \texttt{lsof} & Liste die geöffneten Dateien auf \\
  \rowcolor{Gray}
  \texttt{lspci} & Liste alle PCI-Geräte auf\\
  \texttt{lsusb} & Liste alle USB-Anschlüsse auf\\
  \rowcolor{Gray}
  \texttt{mailq} & Liste den Inhalt der Mailwarteschlange auf \\
  \texttt{reboot} & Starte das System neu\\
  \rowcolor{Gray}
  \texttt{systemctl} & Steuere die Systemdienste \\
  \texttt{systemd-cat} & Verbinde eine Pipe oder eine Programm\-ausgabe mit dem Journal (Logfile) \\
  \rowcolor{Gray}
  \texttt{uname} & Zeige Informationen zum aktuellen Linux\-kernel an \\
  \texttt{uptime} & Gib die Zeitdauer seit dem Systemstart aus \\
  \hline
\end{tabular}

%~\\
\vfill

\section{Finde Dateien und Verzeichnisse}
\begin{tabular}{ p{2.5cm} p{8.5cm} }
  \hline
  \texttt{find} & Finde Dateien und Verzeichnisse anhand \newline verschiedener Kriterien \\
  \rowcolor{Gray}
  \texttt{locate} & Finde Dateien und Verzeichnisse anhand des Namens über die Locate-Datenbank \\
  \texttt{updatedb} & Initialisiere und aktualisiere die Locate-Datenbank \\
  \hline
\end{tabular}

~ \\

\columnbreak

\section{Datei- und Verzeichnisoperationen}
\begin{tabular}{ p{2.5cm} p{8.5cm} }
  \hline
  \texttt{cd} & Wechsele in ein Verzeichnis \\
  \rowcolor{Gray}
  \texttt{cp} & Kopiere eine Datei \\
  \texttt{file} & Identifiziere den Typ einer Datei\\
  \rowcolor{Gray}
  \texttt{ln} & Erzeuge einen Link (Verknüpfung) \\
  \texttt{ls} & Liste den Inhalt eines Verzeichnisses auf \\
  \rowcolor{Gray}
  \texttt{mkdir} & Lege ein Verzeichnis an\\
  \texttt{mmv} & Benenne um oder verschiebe mehrere Dateien und Verzeichnisse \\
  \rowcolor{Gray}
  \texttt{mv} & Benenne um oder verschiebe eine einzelne Datei oder ein Verzeichnis \\
  \texttt{pwd} & Gib das aktuelle Arbeitsverzeichnis aus \\
  \rowcolor{Gray}
  \texttt{rm} & Entferne eine oder mehrere Dateien oder Verzeichnisse \\
  \texttt{rmdir} & Entferne ein leeres Verzeichnis \\
  \rowcolor{Gray}
  \texttt{stat} & Zeige die Details zu einer Datei, einem Verzeichnis oder Link an \\
  \texttt{touch} & Erzeuge eine leere Datei oder aktualisiere den Zeitstempel einer Datei \\
  \rowcolor{Gray}
  \texttt{umask} & Lege die Maske fest, die bei der Erzeugung von Dateien als Vorlage genutzt wird \\
  \hline
\end{tabular}

%~ \\
\hfill

\section{Paketverwaltung (Auswahl)}
\begin{tabular}{ p{2.5cm} p{8.5cm} }
  \hline 
  \texttt{apt}, \texttt{apt-get}, \texttt{aptitude}, \texttt{dpkg}, \texttt{synaptic} & Installiere, aktualisiere, und entferne Software-\newline pakete~\fbox{1} \\
  \rowcolor{Gray}
  \texttt{apt-cache} & Verwalte und befrage den Paketcache~\fbox{1} \\
  \texttt{\small{dpkg-reconfigure}} & Wiederhole die Paketkonfiguration~\fbox{1} \\
  \rowcolor{Gray}
  \texttt{rpm}, \texttt{yum} & Installiere, aktualisiere und entferne Software-\newline pakete~\fbox{2} \\
  \hline
\end{tabular}
~ \\
\noindent Anmerkungen: \\ ~\fbox{1} Debian GNU/Linux, Ubuntu und Linux Mint \\ ~\fbox{2}~RedHat Linux, Fedora und OpenSuse

~ \vfill
\end{multicols}

\newpage

\cheatsheet

\begin{multicols}{2}

\section{Netzwerkkommandos}
\begin{tabular}{ p{2.5cm} p{8.5cm} }
  \hline
  \texttt{dhclient} & Beziehe eine dynamische IP-Adresse \\
  \rowcolor{Gray}
  \texttt{dig}, \texttt{host} & Löse einen Hostnamen über das DNS auf \\
  \texttt{hostname} & Zeige den Hostname des Systems an \\
  \rowcolor{Gray}
  \texttt{ifconfig}, \texttt{iwconfig} & Konfiguriere die Netzwerkschnittstellen und zeige deren Konfiguration an (veraltet)\\
  \texttt{ifdown} & Deaktiviere eine Netzwerkschnittstelle \\
  \rowcolor{Gray}
  \texttt{ifup} & Aktiviere eine Netzwerkschnittstelle\\
  \texttt{ip} & Konfiguriere die Netzwerkschnittstellen und zeige deren Konfiguration an \\
  \rowcolor{Gray}
  \texttt{iwlist} & Liste die verfügbaren WLAN-Netze auf\\
  \texttt{nc}, \texttt{netcat} & Das Schweizer Taschenmesser zu TCP/IP \\
  \rowcolor{Gray}
  \texttt{netstat}, \texttt{ss} & Zeige Netzwerkstatistiken an \\
  \texttt{nmap} & Scanne nach genutzten Ports \\
  \rowcolor{Gray}
  \texttt{nslookup} & Suche in Namensdiensten (\textit{name server}) \\
  \texttt{ping}, \texttt{ping6} & Sende ein ICMP-Paket an einen Zielrechner oder Schnittstelle \\
  \rowcolor{Gray}
  \texttt{route} & Zeige die Routingtabelle an \\
  \texttt{tcpdump} & Gib den Netzwerkverkehr aus \\
  \rowcolor{Gray}
  \texttt{traceroute}, \texttt{traceroute6} & Gib die Route eines Netzwerkpakets zum Zielrechner aus \\
  \texttt{wget} & Nicht-interaktiver Datentransfer von einer Quelle\\
  \hline
\end{tabular}

% \hfill
%~ \\

\section{Sichere Datenübertragung}
\begin{tabular}{ p{2.5cm} p{8.5cm} }
  \hline
  \texttt{rsync}, \texttt{scp} & Synchronisiere Daten und Verzeichnisse lokal und zu einem anderen Rechner über einen verschlüsselten Kanal \\
  \rowcolor{Gray}
  \texttt{ssh} & Verbinde zu einem anderen Rechner über einen verschlüsselten Kanal\\
  \texttt{ssh-copy-id} & Übertrage den öffentlichen Teil eines Schlüsselpaars zu einem entfernten Rechner zur passwortlosen Authentifizierung\\
  \rowcolor{Gray}
  \texttt{ssh-keygen} & Erzeuge ein Schlüsselpaar zur passwortlosen Authentifizierung\\
  \hline
\end{tabular}

~ \\

\columnbreak

\section{Benutzer, Gruppen und Zugangskontrolle}
\begin{tabular}{ p{2.5cm} p{8.5cm} }
  \hline
  \texttt{addgroup} & Füge eine neue Gruppe hinzu~\fbox{1} \\
  \rowcolor{Gray}
  \texttt{adduser} & Füge einen neuen Benutzer hinzu~\fbox{1} \\
  \texttt{chage} & Ändere die Gültigkeit eines Benutzers\\
  \rowcolor{Gray}
  \texttt{chgrp} & Ändere die Gruppe für Datei oder Verzeichnis\\
  \texttt{chmod} & Ändere die Zugriffsrechte für Datei oder Verzeichnis \\
  \rowcolor{Gray}
  \texttt{chown} & Ändere den Eigentümer für Datei oder Verzeichnis\\
  \texttt{delgroup} & Entferne eine bestehende Gruppe~\fbox{1} \\
  \rowcolor{Gray}
  \texttt{deluser} & Entferne einen bestehenden Benutzer~\fbox{1} \\
  \texttt{getent} & Suche die Benutzerinformationen aus den lokalen Datenbanken heraus \\
  \rowcolor{Gray}
  \texttt{getfacl} & Liste die ACLs für Datei oder Verzeichnis auf \\
  \texttt{groupadd} & Lege eine neue Gruppe an~\fbox{2} \\
  \rowcolor{Gray}
  \texttt{groupdel} & Entferne eine bestehende Gruppe~\fbox{2} \\
  \texttt{groupmod} & Ändere eine bestehende Gruppe~\fbox{2} \\
  \rowcolor{Gray}
  \texttt{groups} & Liste die Gruppen auf, zu denen ein Benutzer gehört \\
  \texttt{id} & Zeige die Informationen zu einem Benutzer an \\
  \rowcolor{Gray}
  \texttt{last}, \texttt{lastb} & Zeige den Zeitpunkt der letzten (nicht) erfolgreichen Anmeldung eines Benutzers \\
  \texttt{passwd}, \texttt{usermod} & Ändere das Passwort für einen Benutzer \\
  \rowcolor{Gray}
  \texttt{pwgen} & Generiere ein zufälliges Passwort \\
  \texttt{setfacl} & Setze die ACLs für Datei oder Verzeichnis \\
  \rowcolor{Gray}
  \texttt{su} & Wechsele zu einem anderen Benutzer\\
  \texttt{sudo} & Führe ein Kommando als administrativer Benutzer aus \\
  \rowcolor{Gray}
  \texttt{useradd} & Füge einen neuen Benutzer hinzu~\fbox{2} \\
  \texttt{userdel} & Entferne einen bestehenden Benutzer~\fbox{2} \\
  \rowcolor{Gray}
  \texttt{usermod} & Ändere einen bestehenden Benutzer~\fbox{2} \\
  \texttt{users}, \texttt{w}, \texttt{who} & Liste die angemeldeten Benutzer auf \\
  \rowcolor{Gray}
  \texttt{wall} & Schreibe eine Nachricht an alle gerade angemeldeten Benutzer \\
  \texttt{whoami} & Zeige den aktuellen Benutzernamen an \\
  \hline
\end{tabular}

~ \\
\noindent Anmerkungen: ~\fbox{1} High-Level-Werkzeug ~\fbox{2}~Low-Level-Werkzeug

~ \hfill

\end{multicols}

\newpage

\cheatsheet

\begin{multicols}{2}   

\section{Hilfe bekommen\phantom{g}}
\begin{tabular}{ p{2.5cm} p{8.5cm} }
  \hline
  \texttt{apropos} & Zeige einen kurzen Hilfetext für das Kommando an \\
  \rowcolor{Gray}
  \texttt{info} & Zeige die Hilfe aus dem GNU Info-System an\\
  \texttt{man} & Zeige eine erweiterte Hilfe für das Kommando an \\
  \rowcolor{Gray}
  \texttt{manpath} & Steuere den Pfad zu den Hilfeseiten \\
  \texttt{whatis} & Schaue in der Index-Datenbank für kurze Hilfetexte zu dem angegebenen Kommando nach \\
  \hline
\end{tabular}

~ \\
\vfill

\section{Prozesse und Auftragsverwaltung}
\begin{tabular}{ p{2.5cm} p{8.5cm} }
  \hline
  \texttt{bg} & Sende einen Prozess in den Hintergrund \\
  \rowcolor{Gray}
  \texttt{fg} & Bringe einen Prozess in den Vordergrund \\
  \texttt{htop}, \texttt{top} & Ein Prozessmonitor \\
  \rowcolor{Gray}
  \texttt{jobs} & Liste die laufenden Aufträge auf \\
  \texttt{kill} & Sende ein Signal an einen Prozess \\
  \rowcolor{Gray}
  \texttt{killall} & Sende ein Signal an mehrere Prozesse\\
  \texttt{nice}, \texttt{renice} & Ändere die Priorität eines Prozesses \\
  \rowcolor{Gray}
  \texttt{pgrep} & Finde einen Prozess anhand seines Namens oder eines Regulären Ausdrucks \\
  \texttt{pkill} & Finde Prozesse oder sende ein Signal an sie basierend auf dem Prozessnamen und anderen Attributen\\
  \rowcolor{Gray}
  \texttt{ps} & Liste die laufenden Prozesse auf \\
  \texttt{pstree} & Liste die laufenden Prozesse als Baumstruktur auf \\
  \rowcolor{Gray}
  \texttt{pwdx} & Zeige das aktuelle Arbeitsverzeichnis eines Prozesses an \\
  \texttt{sleep} & Unterbreche (pausiere) die Ausführung eines Kommandos \\
  \rowcolor{Gray}
  \texttt{watch} & Führe ein Kommando periodisch aus und zeige dessen Ausgabe im Vollbildmodus \\
  \hline
\end{tabular}

\section{Verarbeitung von Text und Datenströmen}
\begin{tabular}{ p{2.5cm} p{8.5cm} }
  \hline
  \texttt{awk} & Ändere und extrahiere Daten, bevorzugt spaltenweise \\
  \rowcolor{Gray}
  \texttt{cat} & Gib die Daten vom Anfang bis zum Ende aus\\
  \texttt{cmp}, \texttt{diff} & Vergleiche zwei Dateien und zeige die Unterschiede zwischen beiden an\\
  \rowcolor{Gray}
  \texttt{colrm} & Entferne eine Spalte aus den Daten \\
  \texttt{column} & Gib die Daten spaltenweise aus \\
  \rowcolor{Gray}
  \texttt{cut} &  Extrahiere die Daten spaltenweise\\
  \texttt{egrep}, \texttt{fgrep}, \texttt{grep} & Filtere Daten zeilenweise anhand eines Musters\\
  \rowcolor{Gray}
  \texttt{head} & Gib die ersten Zeilen der Daten aus\\
  \texttt{hexdump} & Gib die Daten in hexadezimaler Schreibweise aus\\
  \rowcolor{Gray}
  \texttt{nl} & Nummeriere die ausgegebenen Zeilen \\
  \texttt{paste} & Verbinde mehrere Dateien zu einer \\
  \rowcolor{Gray}
  \texttt{rev} & Gib Text in umgekehrter Reihenfolge aus \\
  \texttt{sed} & Ändere die Daten zeilenweise \\
  \rowcolor{Gray}
  \texttt{sort} & Sortiere die Daten anhand von Kriterien\\
  \texttt{split} & Teile die Daten anhand von Bedingungen \\
  \rowcolor{Gray}
  \texttt{tac} & Gib die Daten vom Ende bis zum Anfang aus\\
  \texttt{tail} & Gib die letzten Zeilen der Daten aus\\
  \rowcolor{Gray}
  \texttt{tee} & Lies von \texttt{stdin} und gib die Daten in eine Datei und auf \texttt{stdout} aus\\
  \texttt{tr} & Ersetze einzelne Zeichen in den Daten\\
  \rowcolor{Gray}
  \texttt{uniq} & Finde und entferne doppelte Zeilen \\
  \texttt{wc} & Zähle in den Daten die Zeilen, Worte und Zeichen\\
  \hline
\end{tabular}

\section{Umgang mit Logdateien}
\begin{tabular}{ p{2.5cm} p{8.5cm} }
  \hline
  \texttt{journalctl} & Liefere Informationen zu den Logeinträgen via Systemd \\
  \rowcolor{Gray}
  \texttt{logger} & Eine Schnittstelle zu Syslog \\
  \texttt{logrotate} & Rotieren, komprimieren und versenden von  systembezogenen Logdateien \\
  \hline
\end{tabular}

\end{multicols}

\newpage

\cheatsheet

\begin{multicols}{2} 
\section{Umgang mit Dateisystemen}
\begin{tabular}{ p{2.5cm} p{8.5cm} }
  \hline
  \texttt{e2fsck}, \texttt{fsck} & Führe einen Dateisystemcheck durch \\
  \rowcolor{Gray}
  \texttt{findmnt} & Suche und zeige eingebundene Dateisysteme an \\
  \texttt{mkfs} & Erzeuge ein Dateisystem \\
  \rowcolor{Gray}
  \texttt{mkswap} & Bereite eine Partition zur Nutzung als SWAP vor \\
  \texttt{mount} & Binde ein Dateisystem ein \\
  \rowcolor{Gray}
  \texttt{swapoff} & Deaktiviere eine SWAP-Partition \\
  \texttt{swapon} & Aktiviere eine SWAP-Partition \\
  \rowcolor{Gray}
  \texttt{tune2fs} & Ändere ein bestehendes Dateisystem \\
  \texttt{umount} & Hänge ein Dateisystem aus \\
  \hline
\end{tabular}

% ~ \\

\section{Arbeiten mit der Shell, Arbeitsumgebung}
\begin{tabular}{ p{2.5cm} p{8.5cm} }
  \hline
  \texttt{alias} & Auflisten und anpassen der Aliase für Kommmandos \\
  \rowcolor{Gray}
  \texttt{clear} & Lösche den Bildschirm \\
  \texttt{echo} & Gib Text auf dem Bildschirm aus \\
  \rowcolor{Gray}
  \texttt{env} & Zeige die Umgebungsvariablen an \\
  \texttt{exit}, \texttt{logout} & Beende die aktuelle Shell-Sitzung \\
  \rowcolor{Gray}
  \texttt{fc} & Bearbeite die Historie der bisher genutzten Kommandos\\
  \texttt{history} & Zeige die vorher genutzten Kommandos an \\
  \rowcolor{Gray}
  \texttt{screen}, \texttt{tmux} & Ein Terminalmultiplexer \\
  \texttt{set} & Aktiviere oder ändere Laufzeitparameter \\
  \rowcolor{Gray}
  \texttt{time} & Messe die Ausführungszeit eines Kommandos \\
  \texttt{type} & Identifiziere den Kommandotyp \\
  \rowcolor{Gray}
  \texttt{unalias} & Entferne einen Alias für ein Kommando \\
  \texttt{unset} & Entferne oder deaktiviere einen Laufzeitparameter \\
  \rowcolor{Gray}
  \texttt{whatis} & Durchsuche die Indexdatenbank nach Kommando\-beschreibungen \\
  \texttt{whereis} & Suche nach Binärdateien, Quellcode und Handbuchseiten zu einem Kommando \\
  \rowcolor{Gray}
  \texttt{which} & Identifiziere, welches Kommando von der Shell ausgeführt wird\\
  \texttt{xargs} & Stelle Kommandos auf der Basis der Standardeingabe zusammen und führe diese aus \\
  \hline
\end{tabular}

%~\\
%\vfill

\section{Mit dem Linuxkernel umgehen}
\begin{tabular}{ p{2.5cm} p{8.5cm} }
  \hline
  \texttt{dmesg} & Zeige und steuere den Kernel-Ringpuffer\\
  \rowcolor{Gray}
  \texttt{insmod} & Füge ein Kernelmodul in den aktuell laufenden Linux\-kernel ein\\
  \texttt{lsmod} & Liste die Kernelmodule des aktuell laufenden Linux\-kernels auf\\
  \rowcolor{Gray}
  \texttt{modinfo} & Zeige detaillierte Informationen über ein  Kernelmodul an\\
  \texttt{modprobe} & Versuche, ein Kernelmodul zum aktuell laufenden Linux\-kernel hinzuzufügen \\
  \rowcolor{Gray}
  \texttt{rmmod} & Entferne ein Kernelmodul vom aktuell laufenden Linux\-kernel \\
  \texttt{uname} & Zeige Informationen über den aktuell laufenden \newline Linuxkernel an\\
  \hline
\end{tabular}

\section{Drucken}
\begin{tabular}{ p{2.5cm} p{8.5cm} }
  \hline
  \texttt{lp}, \texttt{lpr} & Drucke Dateien oder ändere einen bestehenden Druckauftrag\\
  \rowcolor{Gray}
  \texttt{lpc} & Verwalte die Drucker \\
  \texttt{lpq} & Gib den Inhalt der Druckerwarteschlange aus\\
  \rowcolor{Gray}
  \texttt{lprm} & Entferne einen bestehenden Auftrag aus der Druckerwarteschlange \\
  \hline
\end{tabular}

\section{Systembibliotheken}

\begin{tabular}{ p{2.5cm} p{8.5cm} }
  \hline 
  \texttt{ldd} & Gib die Abhängigkeiten der Shared Librarys aus \\
  \rowcolor{Gray}
  \texttt{ldconfig} & Konfiguriere die Laufzeitbindung des dynamischen Linkers \\
  \hline
\end{tabular}

\section{Benutzen von Pagern}
\begin{tabular}{ p{2.5cm} p{8.5cm} }
  \hline
  \texttt{less}, \texttt{more} & Zeige Daten seitenweise an \\
  \rowcolor{Gray}
  \texttt{most} & Zeige Daten seitenweise oder in mehreren Fenstern an \\
  \hline
\end{tabular}

%\columnbreak

%\vfill


\end{multicols}

\newpage

\cheatsheet

\begin{multicols}{2}

\section{Systemzeit, Datum, Kalender und Lokalisierung}
\begin{tabular}{ p{2.5cm} p{8.5cm} }
  \hline
  \texttt{cal}, \texttt{ncal} & Zeige einen Kalender an\\
  \rowcolor{Gray}
  \texttt{date} & Gib einen Zeitstempel in verschiedenen Formaten aus \\
  \texttt{hwclock} & Steuere sowohl Hardware- als auch die Systemuhr\\
  \rowcolor{Gray}
  \texttt{locale} & Zeige die Informationen zur aktuell genutzten Lokalisierung an \\
  \texttt{ntpdate}, \texttt{ntpsec-ntpdate} & Beziehe die Zeitinformationen von einem NTP-Server \\
  \rowcolor{Gray}
  \texttt{timedatectl} & Steuere die Systemzeit und das Datum \\
  \hline
\end{tabular}

%~ \\
%\vfill

\section{Archive und Datenkomprimierung}
\begin{tabular}{ p{2.5cm} p{8.5cm} }
  \hline 
  \texttt{bunzip2} & Entpacke eine komprimierte Datei (Bzip2-Format) \\
  \rowcolor{Gray}
  \texttt{bzcat} & Gib den Inhalt der mit Bzip2 kompri\-mierten Datei auf \texttt{stdout} aus\\
  \texttt{bzegrep}, \texttt{bzfgrep}, \texttt{bzgrep} & Suche in einer bz-komprimierten Datei \\
  \rowcolor{Gray}
  \texttt{bzip2} & Komprimiere eine Datei (Bzip2-Format) \\
  \texttt{gunzip} & Entpacke eine komprimierte Datei (Gzip-Format) \\
  \rowcolor{Gray}
  \texttt{gzip} & Komprimiere eine Datei (Gzip-Format) \\
  \texttt{tar} & Erzeuge und verwalte ein Bandarchiv (\textit{tar file}) \\
  \rowcolor{Gray}
  \texttt{unzip} & Entpacke eine komprimierte Datei (Zip-Format) \\
  \texttt{xz} & Komprimiere eine Datei (Xz-Format) \\
  \rowcolor{Gray}
  \texttt{xzcat} & Gib den Inhalt der mit Xz komprimierten Datei auf \texttt{stdout} aus \\
  \texttt{xzegrep}, \texttt{xzfgrep}, \texttt{xzgrep} & Suche in einer xz-komprimierten Datei \\
  \rowcolor{Gray}
  \texttt{zip} & Komprimiere eine Datei (Zip-Format) \\
  \texttt{zcat} & Gib den Inhalt der mit Gzip kompri\-mierten Datei auf \texttt{stdout} aus \\
  \rowcolor{Gray}
  \texttt{zegrep}, \texttt{zfgrep}, \texttt{zgrep} & Suche in einer zip-komprimierten Datei \\
  \hline
\end{tabular}


\columnbreak

\section*{Notizen\phantom{g}}

\begin{tabular}{ p{2.5cm} p{8.5cm} }
  \hline
  ~ & \\
  \rowcolor{Gray}
  ~ & \\
  ~ & \\
  \rowcolor{Gray}
  ~ & \\
  ~ & \\
  \rowcolor{Gray}
  ~ & \\
  ~ & \\
  \rowcolor{Gray}
  ~ & \\
  ~ & \\
  \rowcolor{Gray}
  ~ & \\
  ~ & \\
  \rowcolor{Gray}
  ~ & \\
  ~ & \\
  \rowcolor{Gray}
  ~ & \\
  ~ & \\
  \rowcolor{Gray}
  ~ & \\
  ~ & \\
  \rowcolor{Gray}
  ~ & \\
  ~ & \\
  \rowcolor{Gray}
  ~ & \\
  ~ & \\
  \rowcolor{Gray}
  ~ & \\
  ~ & \\
  \rowcolor{Gray}
  ~ & \\
  ~ & \\
  \rowcolor{Gray}
  ~ & \\
  ~ & \\
  \rowcolor{Gray}
  ~ & \\
  ~ & \\
  \rowcolor{Gray}
  ~ & \\
  ~ & \\
  \rowcolor{Gray}
  ~ & \\
  ~ & \\
  \rowcolor{Gray}
  ~ & \\
  \hline
\end{tabular}
\end{multicols}

\end{document}
