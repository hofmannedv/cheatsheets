% Debian Paketmanagement
% Version 0.4
% (C) 2023 Frank Hofmann <info@efho.de>
% Available from https://github.com/hofmannedv/cheatsheets

% Published under Creative Commons CC-BY-SA 4.0 International License
% https://creativecommons.org/licenses/by-sa/4.0/

\documentclass[10pt]{article}
\usepackage{cheatsheet}

\renewcommand{\cheatsheetTitle}{Cheatsheet für das Debian-Paketmanagement}
\renewcommand{\cheatsheetVersion}{Version 0.4}
\renewcommand{\cheatsheetCopyright}{
  \copyright~ 2023 Frank Hofmann \faIcon{envelope} \href{mailto:info@efho.de}{info@efho.de} \\ Veröffentlicht unter 
 \href{https://creativecommons.org/licenses/by-sa/4.0/}{Creative Commons CC-BY-SA 4.0 \\ International License}. Erstellt mit \LaTeX. \\ \faIcon{github} \href{https://github.com/hofmannedv/cheatsheets}{https://github.com/hofmannedv/cheatsheets}~.
}
\begin{document}

\cheatsheet

\section{Die lokale Paketliste aktualisieren}

\begin{tabular}{ p{3.5cm} p{9cm} p{11cm}}
  \hline
  \rowcolor{Gray}
  \textbf{Werkzeug} & \textbf{Kommandofolge} & \textbf{Anmerkungen} \\
  \hline 
  \textbf{APT} & \texttt{apt update} & \\
  \rowcolor{Gray}
  & \texttt{apt-get update} & \\
  \textbf{Aptitude} & \texttt{aptitude update} & \\
  \hline 
\end{tabular}

\section{Neue Softwarepakete installieren}
\begin{tabular}{ p{3.5cm} p{9cm} p{11cm}}
  \hline
  \rowcolor{Gray}
  \textbf{Werkzeug} & \textbf{Kommandofolge} & \textbf{Anmerkungen} \\
  \hline 
  \textbf{APT} & \texttt{apt install \textit{Paketname}} & Das Paket wird vom Paketrepository bezogen \\
  \rowcolor{Gray}
   & \texttt{apt --no-download install \textit{Paketname}} & Das Paket wird nicht vom Paketrepository bezogen, sondern statt\-dessen aus dem Paketcache entnommen \\ 
   & \texttt{apt-get install \textit{Paketname}} & Das Paket wird vom Paketrepository bezogen \\
  \rowcolor{Gray}
   & \texttt{apt-get --no-download install \textit{Paketname}} & Das Paket wird nicht vom Paketrepository bezogen, sondern statt\-dessen aus dem Paketcache entnommen \\
  \textbf{Aptitude} & \texttt{aptitude install \textit{Paketname}} & Das Paket wird vom Paketrepository bezogen \\
  \rowcolor{Gray}
  \textbf{Dpkg} & \texttt{dpkg -i \textit{Dateiname}} & Installiere das Paket als \texttt{.deb}-Datei aus dem aktuellen Verzeichnis \\
  \hline
\end{tabular}

\section{Bereits bestehende Softwarepakete erneuern}
\begin{tabular}{ p{3.5cm} p{9cm} p{11cm}}
  \hline
  \rowcolor{Gray}
  \textbf{Werkzeug} & \textbf{Kommandofolge} & \textbf{Anmerkungen} \\
  \hline 
  \textbf{APT} & \texttt{apt install \textit{Paketname}} & Ein Paket wird aktualisiert, sofern dieses bereits installiert ist. Das Paket wird vom Paketrepository bezogen. \\
  \rowcolor{Gray}
   & \texttt{apt safe-upgrade \textit{Paketname}} &  Eine neuere Version des Paketes wird installiert. Das neue Paket wird vom Paketrepository bezogen.\\
   & \texttt{apt-get install \textit{Paketname}} & Ein Paket wird aktualisiert, sofern dieses bereits installiert ist. Das neue Paket wird vom Paketrepository bezogen. \\
  \rowcolor{Gray}
   & \texttt{apt-get upgrade \textit{Paketname}} & Eine neuere Version des Paketes wird installiert. Das Paket wird vom Paketrepository bezogen.\\
  \textbf{Aptitude} & \texttt{aptitude safe-upgrade \textit{Paketname}} &  Eine neuere Version des Paketes wird installiert. Das Paket wird vom Paketrepository bezogen.\\
  \hline
\end{tabular}

\newpage

\cheatsheet

\section{Ein Softwarepaket erneut installieren}
\begin{tabular}{ p{3.5cm} p{9cm} p{11cm}}
  \hline
  \rowcolor{Gray}
  \textbf{Werkzeug} & \textbf{Kommandofolge} & \textbf{Anmerkungen} \\
  \hline 
  \textbf{APT} & \texttt{apt reinstall \textit{Paketname}} & Das Paket wird vom Paketrepository bezogen \\
  \rowcolor{Gray}
  & \texttt{apt-get install --reinstall \textit{Paketname}} & Das Paket wird vom Paketrepository bezogen \\
  & \texttt{apt-get reinstall \textit{Paketname}} & Das Paket wird vom Paketrepository bezogen \\
  \rowcolor{Gray}
  \textbf{Aptitude} & \texttt{aptitude reinstall \textit{Paketname}} & Das Paket wird vom Paketrepository bezogen \\
  \hline
\end{tabular}

\section{Softwarepakete nur herunterladen}
\begin{tabular}{ p{3.5cm} p{9cm} p{11cm}}
  \hline
  \rowcolor{Gray}
  \textbf{Werkzeug} & \textbf{Kommandofolge} & \textbf{Anmerkungen} \\
  \hline 
  \textbf{APT} & \texttt{apt download \textit{Paketname}} & Das Paket wird vom Paketrepository bezogen und im aktuellen \newline Verzeichnis gespeichert. Es kann nachfolgend mit \texttt{dpkg -i \textit{Dateiname}} installiert werden\\
  \rowcolor{Gray}
  \textbf{Aptitude} & \texttt{aptitude download \textit{Paketname}} & Das Paket wird vom Paketrepository bezogen und im aktuellen \newline Verzeichnis gespeichert. Es kann nachfolgend mit \texttt{dpkg -i \textit{Dateiname}} installiert werden \\
  \hline
\end{tabular}

\section{Softwarepakete entfernen und die Konfigurationsdateien behalten}
\begin{tabular}{ p{3.5cm} p{9cm} p{11cm}}
  \hline
  \rowcolor{Gray}
  \textbf{Werkzeug} & \textbf{Kommandofolge} & \textbf{Anmerkungen} \\
  \hline 
  \textbf{APT} & \texttt{apt remove \textit{Paketname}} & \\
  \rowcolor{Gray}
  & \texttt{apt-get remove \textit{Paketname}} & \\
  \textbf{Aptitude} & \texttt{aptitude remove \textit{Paketname}} & \\
  \rowcolor{Gray}
  \textbf{Dpkg} & \texttt{dpkg -r \textit{Paketname}} & \\
  & \texttt{dpkg --remove \textit{Paketname}} & \\
  \hline
\end{tabular}

\section{Softwarepakete samt Konfigurationsdateien entfernen}
\begin{tabular}{ p{3.5cm} p{9cm} p{11cm}}
  \hline
  \rowcolor{Gray}
  \textbf{Werkzeug} & \textbf{Kommandofolge} & \textbf{Anmerkungen} \\
  \hline 
  \textbf{APT}& \texttt{apt purge \textit{Paketname}} & \\
  \rowcolor{Gray}
  & \texttt{apt-get purge \textit{Paketname}} & \\
  & \texttt{apt-get --purge remove \textit{Paketname}} & \\
  \rowcolor{Gray}
  \textbf{Aptitude} & \texttt{aptitude purge \textit{Paketname}} & \\
  \textbf{Dpkg} & \texttt{dpkg -P \textit{Paketname}} & \\
  \rowcolor{Gray}
  & \texttt{dpkg --purge \textit{Paketname}} & \\
  \hline
\end{tabular}

\newpage

\cheatsheet

\section{Installierte Softwarepakete auflisten}
\begin{tabular}{ p{3.5cm} p{9cm} p{11cm}}
  \hline
  \rowcolor{Gray}
  \textbf{Werkzeug} & \textbf{Kommandofolge} & \textbf{Anmerkungen} \\
  \hline 
  \textbf{APT}& \texttt{apt list --installed} & Liste alle installierten Pakete auf\\
  \rowcolor{Gray}
  \textbf{Apt-cache} & \texttt{apt-cache search \textit{Paketname}} & Liste alle verfügbaren Pakete auf, die zu dem Paketnamen passen; die Suche umfaßt den Paketnamen, als auch die Paketbeschreibung \\
  & \texttt{apt-cache search --names-only \textit{Paketname}} & Liste alle verfügbaren Pakete auf, die zu dem Paketnamen passen; die Suche umfaßt nur den Paketnamen \\
  \rowcolor{Gray}
  \textbf{Aptitude} & \texttt{aptitude search '\textasciitilde{}i'} & Liste alle installierten Pakete auf \\
  \textbf{Dpkg} & \texttt{dpkg -l} & Liste alle installierten Pakete auf \\
  \rowcolor{Gray}
  & \texttt{dpkg --list} & Liste alle installierten Pakete auf \\
  \hline
\end{tabular}

\section{Information zu einem einzelnen Softwarepaket anzeigen}
\begin{tabular}{ p{3.5cm} p{9cm} p{11cm}}
  \hline
  \rowcolor{Gray}
  \textbf{Werkzeug} & \textbf{Kommandofolge} & \textbf{Anmerkungen} \\
  \hline 
  \textbf{APT} & \texttt{apt list \textit{Paketname}} & Zeige die Paketinformation und den Installationsstatus an \\
  \rowcolor{Gray}
  & \texttt{apt show \textit{Paketname}} & Zeige die detaillierte Paketinformation und den Installationsstatus an \\
  \textbf{Aptitude} & \texttt{aptitude show \textit{Paketname}} & Zeige die detaillierte Paketinformation und den Installationsstatus an \\
  \textbf{Dpkg} & \texttt{dpkg -l \textit{Paketname} } & Zeige den Installationsstatus an \\
  \rowcolor{Gray}
  & \texttt{dpkg --list \textit{Paketname} } & Zeige den Installationsstatus an\\
  & \texttt{dpkg -s \textit{Paketname} } & Zeige die Paketinformation an\\
  \rowcolor{Gray}
  & \texttt{dpkg --status \textit{Paketname} } & Zeige die Paketinformation an\\
  \hline
\end{tabular}

\section{Paketabhängigkeiten anzeigen}
\begin{tabular}{ p{3.5cm} p{9cm} p{11cm}}
  \hline
  \rowcolor{Gray}
  \textbf{Werkzeug} & \textbf{Kommandofolge} & \textbf{Anmerkungen} \\
  \hline 
  \textbf{Apt-cache} & \texttt{apt-cache depends \textit{Paketname}} & Zeige die umgekehrten Abhängigkeiten an \\
  \rowcolor{Gray}
  & \texttt{apt-cache rdepends \textit{Paketname}} & Zeige die Abhängigkeiten an \\
  \textbf{Aptitude} & \texttt{aptitude search '\textasciitilde{R}' \textit{Paketname}} & Zeige die Abhängigkeiten an \\
  \rowcolor{Gray}
  & \texttt{aptitude search '\textasciitilde{D}' \textit{Paketname}} & Zeige die umgekehrten Abhängigkeiten an\\
  \textbf{Apt-rdepends} & \texttt{apt-rdepends -r \textit{Paketname}} & Zeige die Abhängigkeiten an \\
  \rowcolor{Gray}
  & \texttt{apt-rdepends \textit{Paketname}} & Zeige die umgekehrten Abhängigkeiten an \\
  \textbf{Dpkg} & \texttt{dpkg -f \textit{Dateiname}} Depends & Zeige die Abhängigkeiten an; der Dateiname bezieht sich auf eine lokale \texttt{.deb}-Datei \\
  \rowcolor{Gray}
  & \texttt{dpkg-deb -f \textit{Dateiname}} Depends & Zeige die Abhängigkeiten an; der Dateiname bezieht sich auf eine lokale \texttt{.deb}-Datei \\
  \textbf{Grep-status} & \texttt{grep-status -F Paketname -s Depends \textit{Paketname}} & Zeige die Abhängigkeiten an \\
  \rowcolor{Gray}
  & \texttt{grep-status -P -s Depends \textit{Paketname}} & Zeige die Abhängigkeiten an \\
  \hline
\end{tabular}

\newpage

\cheatsheet

\section{Dateien eines Pakets vor der Installation anzeigen}
\begin{tabular}{ p{3.5cm} p{9cm} p{11cm}}
  \hline
  \rowcolor{Gray}
  \textbf{Werkzeug} & \textbf{Kommandofolge} & \textbf{Anmerkungen} \\
  \hline 
  \textbf{Dpkg} & \texttt{dpkg -c \textit{Paketname}} & \\
  \rowcolor{Gray}
  & \texttt{dpkg --contents \textit{Paketname}} & \\
  \textbf{Dpkg-deb} & \texttt{dpkg-deb -c \textit{Paketname}} & \\
  \rowcolor{Gray}
  & \texttt{dpkg-deb --contents \textit{Paketname}} & \\
  \hline
\end{tabular}

\section{Dateien eines installierten Pakets auflisten}
\begin{tabular}{ p{3.5cm} p{9cm} p{11cm}}
  \hline
  \rowcolor{Gray}
  \textbf{Werkzeug} & \textbf{Kommandofolge} & \textbf{Anmerkungen} \\
  \hline 
  \textbf{Apt-file} & \texttt{apt-file -list \textit{Paketname}} & \\
  \rowcolor{Gray}
  & \texttt{apt-file -show \textit{Paketname}} & \\
  \textbf{Dpkg} & \texttt{dpkg -L \textit{Paketname}} & \\
  \rowcolor{Gray}
  & \texttt{dpkg --listfiles \textit{Paketname}} & \\
  \textbf{Dpkg-query} & \texttt{dpkg-query -L \textit{Paketname}} & \\
  \rowcolor{Gray}
  & \texttt{dpkg-query --listfiles \textit{Paketname}} & \\
  \hline
\end{tabular}

\section{Konfigurationsdateien eines Pakets auflisten}
\begin{tabular}{ p{3.5cm} p{9cm} p{11cm}}
  \hline
  \rowcolor{Gray}
  \textbf{Werkzeug} & \textbf{Kommandofolge} & \textbf{Anmerkungen} \\
  \hline 
  \textbf{Dpkg} & \texttt{dpkg -s \textit{Paketname}} & \\
  \rowcolor{Gray}
  & \texttt{dpkg --status \textit{Paketname}} & \\
  \textbf{Dpkg-query} & \texttt{dpkg-query -s \textit{Paketname}} & \\
  \rowcolor{Gray}
  & \texttt{dpkg-query --status \textit{Paketname}} & \\
  \hline
\end{tabular}

\section{Das Paket identifizieren, aus dem eine Datei stammt}
\begin{tabular}{ p{3.5cm} p{9cm} p{11cm}}
  \hline
  \rowcolor{Gray}
  \textbf{Werkzeug} & \textbf{Kommandofolge} & \textbf{Anmerkungen} \\
  \hline 
  \textbf{Apt-file} & \texttt{apt-file find \textit{Dateiname}} & Suche in allen verfügbaren Paketen\\
  \rowcolor{Gray}
  & \texttt{apt-file search \textit{Dateiname}} & Suche in allen verfügbaren Paketen\\
  \textbf{Dpkg} & \texttt{dpkg -S \textit{Dateiname}} & Suche in allen installierten Paketen\\
  \rowcolor{Gray}
  & \texttt{dpkg --search \textit{Dateiname}} & Suche in allen installierten Paketen \\
  \textbf{Dpkg-query} & \texttt{dpkg-query -S \textit{Dateiname}} & Suche in allen installierten Paketen\\
  \rowcolor{Gray}
  & \texttt{dpkg-query --search \textit{Dateiname}} & Suche in allen installierten Paketen \\
  \hline
\end{tabular}

\newpage
\cheatsheet

\section{Den Paketstatus anzeigen}
\begin{tabular}{ p{3.5cm} p{9cm} p{11cm}}
  \hline
  \rowcolor{Gray}
  \textbf{Werkzeug} & \textbf{Kommandofolge} & \textbf{Anmerkungen} \\
  \hline 
  \textbf{Apt-cache} & \texttt{apt-cache show \textit{Paketname}} & Suche in allen verfügbaren Paketen\\
  \rowcolor{Gray}
  \textbf{Aptitude} & \texttt{aptitude show \textit{Paketname}} & Suche in allen verfügbaren Paketen\\
  \textbf{Dpkg} & \texttt{dpkg -s \textit{Paketname}} & Das Paket muß lokal installiert sein \\
  \rowcolor{Gray}
  & \texttt{dpkg --status \textit{Paketname}} & Das Paket muß lokal installiert sein\\
  & \texttt{dpkg -I \textit{Dateiname}} & Der Dateiname bezieht sich auf eine lokale \texttt{.deb}-Datei \\
  \rowcolor{Gray}
  & \texttt{dpkg --info \textit{Dateiname}} & Der Dateiname bezieht sich auf eine lokale \texttt{.deb}-Datei\\
  \textbf{Dpkg-query} & \texttt{dpkg-query -s \textit{Paketname}} & Das Paket muß lokal installiert sein\\
  \rowcolor{Gray}
  & \texttt{dpkg-query --status \textit{Paketname}} & Das Paket muß lokal installiert sein \\
  \hline
\end{tabular}

\section{Die Paketsignatur überprüfen}
\begin{tabular}{ p{3.5cm} p{9cm} p{11cm}}
  \hline
  \rowcolor{Gray}
  \textbf{Werkzeug} & \textbf{Kommandofolge} & \textbf{Anmerkungen} \\
  \hline 
  \textbf{Debsums} & \texttt{debsums \textit{Paketname}} & Überprüfe alle Dateien des Pakets mit Ausnahme der Konfigurationsdateien\\
  \rowcolor{Gray}
  \textbf{Debsums} & \texttt{debsums} & Überprüfe alle Dateien aller Pakete mit Ausnahme der Konfigurationsdateien \\
  \textbf{Dpkg} & \texttt{dpkg -V \textit{Paketname}} & Überprüfe alle Dateien des Pakets \\
  \rowcolor{Gray}
  & \texttt{dpkg --verify \textit{Paketname}} & Überprüfe alle Dateien des Pakets \\
  \textbf{Dpkg-sig} & \texttt{dpkg-sig --verify \textit{Paketname}} & Überprüfe die GPG-Signatur des Pakets \\
  \rowcolor{Gray}
  \textbf{Gpg} & \texttt{gpg --verify \textit{Paketname}} & Überprüfe die GPG-Signatur des Pakets \\
  \hline
\end{tabular}

\section{Ein bereits installiertes Paket auf Änderungen überprüfen}
\begin{tabular}{ p{3.5cm} p{9cm} p{11cm}}
  \hline
  \rowcolor{Gray}
  \textbf{Werkzeug} & \textbf{Kommandofolge} & \textbf{Anmerkungen} \\
  \hline 
  \textbf{Dpkg} & \texttt{dpkg --verify \textit{Paketname}} & Prüfe nur dieses Paket \\
  \rowcolor{Gray}
  & \texttt{dpkg --verify} & Prüfe alle Pakete \\
  \hline
\end{tabular}

\newpage

\cheatsheet

\section{Verfügbare Pakete auflisten}
\begin{tabular}{ p{3.5cm} p{9cm} p{11cm}}
  \hline
  \rowcolor{Gray}
  \textbf{Werkzeug} & \textbf{Kommandofolge} & \textbf{Anmerkungen} \\
  \hline 
  \textbf{APT} & \texttt{apt list} & Liste alle bekannten und verfügbaren Pakete mit deren Status auf\\
  \rowcolor{Gray}
  & \texttt{apt list \textit{Paketname}} & Liste alle bekannten und verfügbaren Versionen des Pakets mit dessen Status auf\\
  \textbf{Apt-cache} & \texttt{apt-cache pkgnames} & Liste alle bekannten und verfügbaren Pakete auf\\
  \hline
\end{tabular}

\section{Mit neuer Version aktualisierbare Softwarepakete anzeigen}
\begin{tabular}{ p{3.5cm} p{9cm} p{11cm}}
  \hline
  \rowcolor{Gray}
  \textbf{Werkzeug} & \textbf{Kommandofolge} & \textbf{Anmerkungen} \\
  \hline 
  \textbf{APT} & \texttt{apt list --upgradable} & Liste die verfügbaren Pakete mit der neuen Versionsnummer auf\\
  \rowcolor{Gray}
  & \texttt{apt list --upgradable \textit{Paketname}} & Liste die neuen Versionen für dieses Paket auf\\
  & \texttt{apt-get upgrade -u} & Liste die verfügbaren Pakete mit der neuen Versionsnummer auf \\
  \rowcolor{Gray}
  & \texttt{apt-get upgrade --show-upgraded} & Liste die verfügbaren Pakete mit der neuen Versionsnummer auf \\
  & \texttt{apt-get upgrade -u -s} & Simuliere die Aktualisierung der verfügbaren Pakete mit einer neuen Version \\
  \rowcolor{Gray}
  & \texttt{apt-get upgrade --show-upgraded --simulate} & Simuliere die Aktualisierung der verfügbaren Pakete mit einer neuen Version \\
  \textbf{Aptitude} & \texttt{aptitude search '\textasciitilde{U}'} & Liste die verfügbaren Pakete mit der neuen Versionsnummer auf \\
  \rowcolor{Gray}
  & \texttt{aptitude search ?upgradable} &  Liste die verfügbaren Pakete mit der neuen Versionsnummer auf \\
  \hline
\end{tabular}

\section{Eine installierte Veröffentlichung aktualisieren (Distributionsupgrade)}
\begin{tabular}{ p{3.5cm} p{9cm} p{11cm}}
  \hline
  \rowcolor{Gray}
  \textbf{Werkzeug} & \textbf{Kommandofolge} & \textbf{Anmerkungen} \\
  \hline 
  \textbf{APT} & \texttt{apt full-upgrade} & Aktualisiere alle bereits installierten Softwarepakete auf die aktuellste, verfügbare Version \\
  \rowcolor{Gray}
  & \texttt{apt-get dist-upgrade} & Aktualisiere alle bereits installierten Softwarepakete auf die aktuellste, verfügbare Version \\
  \textbf{Aptitude} & \texttt{aptitude full-upgrade} & Aktualisiere alle bereits installierten Softwarepakete auf die aktuellste, verfügbare Version \\
  \rowcolor{Gray}
  & \texttt{aptitude safe-upgrade} &  Aktualisiere alle bereits installierten Softwarepakete auf die aktuellste, verfügbare Version mit der Ausnahme von ungenutzten Paketen \\
  \hline
\end{tabular}

\newpage

\cheatsheet

\section{Empfohlene Bücher und Materialien}

\begin{description}
    \item[Axel Beckert, Frank Hofmann: Das Debian-Paketmanagement-Buch] ~ \\
        \begin{tabbing}
            \= Verfügbare Sprachen: \= \= Deutsch \\
            \> Projektwebseite: \> \> \href{https://dpmb.org/}{https://dpmb.org/} \\
            \> Mitwirkung: \> \> via GitHub ~ \faIcon{github} \href{https://github.com/dpmb}{https://github.com/dpmb} \\
            \> Debianpaket: \> \> \href{https://packages.debian.org/stable/debian-paketmanagement-buch}{\texttt{debian-paketmanagement-buch}} \\ 
        \end{tabbing}

    \item[Raphaël Hertzog, Roland Mas: The Debian Administrator's Handbook] ~ \\ 
        \begin{tabbing}
            \= Verfügbare Sprachen: \= \= Französisch, Englisch, Deutsch, Russisch, Japanisch und andere Sprachen \\
            \> Projektwebseite \> \> \href{https://debian-handbook.info/}{https://debian-handbook.info/} \\
            \> Mitwirkung: \> \> via Salsa ~ \href{https://salsa.debian.org/hertzog/debian-handbook/}{https://salsa.debian.org/hertzog/debian-handbook/} \\
            \> Debianpaket: \> \> \href{https://packages.debian.org/stable/debian-handbook}{\texttt{debian-handbook}} \\
        \end{tabbing}

    \item[Frank Ronneburg: Das Debian GNU/Linux Anwenderhandbuch] ~ \\
        \begin{tabbing}
            \= Verfügbare Sprachen: \= \= Deutsch \\
            \> Projektwebseite: \> \> \href{https://debiananwenderhandbuch.de/}{https://debiananwenderhandbuch.de/} \\
            \> Mitwirkung: \> \> via Email ~ \faIcon{envelope} 
            \href{mailto:fr@debiananwenderhandbuch.de}{fr@debiananwenderhandbuch.de} \\
            
            \> Debianpaket: \> \> nicht vorhanden \\
        \end{tabbing}
        
    \item[Matthew Danish: Die apt-dpkg-Referenzliste] ~ \\
        \begin{tabbing}
            \= Verfügbare Sprachen: \= \= Englisch \\
            \> Projektwebseite: \> \> nicht vorhanden \\
            \> Mitwirkung: \> \> nicht bekannt \\
            
            \> Debianpaket: \> \> \texttt{apt-dpkg-ref} (verfügbar bis Debian 11 \textit{Bullseye})
        \end{tabbing}
\end{description}

\end{document}
