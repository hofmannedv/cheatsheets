% Red Hat Package Management
% Version 0.3
% (C) 2023,2024 Frank Hofmann <info@efho.de>
% Available from https://github.com/hofmannedv/cheatsheets

% Published under Creative Commons CC-BY-SA 4.0 International License
% https://creativecommons.org/licenses/by-sa/4.0/

\documentclass[10pt]{article}
\usepackage{cheatsheet}

\renewcommand{\cheatsheetTitle}{Red Hat Package \newline Management Cheatsheet}
\renewcommand{\cheatsheetVersion}{Version 0.3}
\renewcommand{\cheatsheetCopyright}{
  \copyright~ 2023,2024 by Frank Hofmann \\ \faIcon{envelope} \href{mailto:info@efho.de}{info@efho.de}. Published under 
 \href{https://creativecommons.org/licenses/by-sa/4.0/}{Creative Commons \\ CC-BY-SA 4.0 International License}. Created with \LaTeX. \\ \faIcon{github} \href{https://github.com/hofmannedv/cheatsheets}{https://github.com/hofmannedv/cheatsheets}~.
}

% add document metadata as XMP data
% add keywords based on hyperref package
\usepackage{hyperxmp}
\hypersetup{
    pdfauthor={Frank Hofmann}, 
    pdftitle={Red Hat Package Management Cheatsheet},
    pdfkeywords={Package management, Red Hat, rpm, yum},
    pdfsubject={Commands for handling software packages effectively on Red Hat Linux},
    pdfcopyright={Copyright (C) 2023,2024 by Frank Hofmann. Published under Creative Commons Attribution - Share Alike 4.0 International License (CC-BY-SA-4.0)}
    pdfcopyrighturl={https://creativecommons.org/licenses/by-sa/4.0/legalcode.en}
}

\begin{document}

\cheatsheet

%\section{Update the local package list}

\section{Install new software packages}
\begin{tabular}{ p{3.5cm} p{9cm} p{11cm}}
  \hline
  \rowcolor{Gray}
  \textbf{Tool} & \textbf{Command} & \textbf{Notes} \\
  \hline 
  \textbf{RPM} & \texttt{rpm -i \textit{package file}} \newline \texttt{rpm --install \textit{package file}} & Package file refers  to a local \texttt{.rpm} file in the current directory\\
  \rowcolor{Gray}
  & \texttt{rpm -i --nodeps \textit{package file}} \newline \texttt{rpm --install --nodeps \textit{package file}} & Package file refers  to a local \texttt{.rpm} file in the current directory; ignore package dependencies\\
  & \texttt{rpm -ihv \textit{package file}} & Package file refers  to a local \texttt{.rpm} file in the current directory; installation with verbose output, and process bar\\
  \rowcolor{Gray}
  \textbf{YUM} & \texttt{yum groupinstall \textit{package group}} & Install all packages from a package group from a software repository\\
  & \texttt{yum install \textit{package}} & Install a package from a software repository\\
  \rowcolor{Gray}
  & \texttt{yum localinstall \textit{package}} & Install a package from a local file, http, or ftp resource\\
  & \texttt{yum reinstall \textit{package}} & Reinstall the current version of a package from a software repository\\
  \hline
\end{tabular}

\section{Renew existing software packages}
\begin{tabular}{ p{3.5cm} p{9cm} p{11cm}}
  \hline
  \rowcolor{Gray}
  \textbf{Tool} & \textbf{Command} & \textbf{Notes} \\
  \hline 
  \textbf{RPM} & \texttt{rpm -F \textit{package file}} \newline \texttt{rpm --freshen \textit{package file}} & Upgrade an already installed package; based on a local file. \\
  \rowcolor{Gray}
  & \texttt{rpm -U \textit{package file}} \newline \texttt{rpm --upgrade \textit{package file}} & Upgrade an already installed package, or install package if not already installed; based on a local file. \\
  \textbf{YUM} & \texttt{yum update} & Update all packages with available updates; based on a software repository \\
  \rowcolor{Gray}
  & \texttt{yum update \textit{package}} & Update the specified package, only; based on a software repository \\  
  & \texttt{yum update --security} & Apply security-related package updates, only; based on a software repository \\  
  \rowcolor{Gray}
  & \texttt{yum update-to \textit{package-version}} & Update the specified package, or all packages up to a specific version; based on a software repository \\  
  & \texttt{yum upgrade} & Update all installed packages taking obsoletes into account; based on a software repository \\  
  \rowcolor{Gray}
  & \texttt{yum upgrade \textit{package}} & Update the specified package taking obsoletes into account; based on a software repository \\  
  \hline
\end{tabular}

\newpage

\cheatsheet

\section{Reinstall an existing software package}
\begin{tabular}{ p{3.5cm} p{9cm} p{11cm}}
  \hline
  \rowcolor{Gray}
  \textbf{Tool} & \textbf{Command} & \textbf{Notes} \\
  \hline 
  \textbf{RPM} & \texttt{rpm --reinstall \textit{package file}} & Package file refers to a local file \\
  \rowcolor{Gray}
  \textbf{YUM} & \texttt{yum reinstall \textit{package}} & Package is obtained from the package repository \\
  \hline
\end{tabular}

\section{Install a previous version of a software package (downgrade)}
\begin{tabular}{ p{3.5cm} p{9cm} p{11cm}}
  \hline
  \rowcolor{Gray}
  \textbf{Tool} & \textbf{Command} & \textbf{Notes} \\
  \hline 
  \textbf{YUM} & \texttt{yum downgrade \textit{package}} & The previous version of the package is installed \\
  \hline
\end{tabular}

\section{Download software packages, only}
\begin{tabular}{ p{3.5cm} p{9cm} p{11cm}}
  \hline
  \rowcolor{Gray}
  \textbf{Tool} & \textbf{Command} & \textbf{Notes} \\
  \hline 
  \textbf{YUM} & \texttt{yum install --downloadonly \textit{package}} & The package is obtained from the package repository, and stored in the package cache. After that it can be installed using the command \newline \texttt{rpm -i \textit{filename}}~.\\
  \rowcolor{Gray}
  & \texttt{yumdownloader \textit{package}} & The package is obtained from the package repository, and stored in the current directory. After that it can be installed using the command \newline \texttt{rpm -i \textit{filename}}~.\\
  & \texttt{yumdownloader --resolve \textit{package}} & The package, and its dependencies are obtained from the package repository, and are stored in the current directory. After that these packages can be installed using the command \texttt{rpm -i \textit{filename}}~.\\
  \hline
\end{tabular}

\section{Remove software packages as well as the configuration files}
\begin{tabular}{ p{3.5cm} p{9cm} p{11cm}}
  \hline
  \rowcolor{Gray}
  \textbf{Tool} & \textbf{Command} & \textbf{Notes} \\
  \hline 
  \textbf{RPM} & \texttt{rpm -e \textit{package}} \newline \texttt{rpm --erase \textit{package}}& \\
  \rowcolor{Gray}
  & \texttt{rpm -e --nodeps \textit{package}} \newline \texttt{rpm --erase  --nodeps\textit{package}} & Remove the package without a dependency check\\
  \textbf{YUM} & \texttt{yum erase \textit{package}} \newline \texttt{yum remove \textit{package}} & Remove the package (and possibly dependencies)\\
  \rowcolor{Gray}
  \hline
\end{tabular}

\newpage

\cheatsheet

\section{List installed software packages}
\begin{tabular}{ p{3.5cm} p{9cm} p{11cm}}
  \hline
  \rowcolor{Gray}
  \textbf{Tool} & \textbf{Command} & \textbf{Notes} \\
  \hline 
  \textbf{RPM} & \texttt{rpm -qa} \newline \texttt{rpm --query --all} & List all installed packages\\
  \rowcolor{Gray}
  & \texttt{rpm -qa --last} \newline \texttt{rpm --query --all --last} & List all installed packages, display the most recent ones first\\
  \textbf{YUM} & \texttt{yum list installed} & List all installed packages\\
  \rowcolor{Gray}
  & \texttt{yum list all} & List all installed, and available packages\\
  & \texttt{yum list \textit{package}} & Show installed version for this package\\
  \hline
\end{tabular}

\section{Display information about a single software package}
\begin{tabular}{ p{3.5cm} p{9cm} p{11cm}}
  \hline
  \rowcolor{Gray}
  \textbf{Tool} & \textbf{Command} & \textbf{Notes} \\
  \hline 
  \textbf{RPM} & \texttt{rpm -q \textit{package}} \newline \texttt{rpm --query \textit{package}} & Display installation status for the package \\
  \rowcolor{Gray}
  & \texttt{rpm -qp \textit{package file}} \newline \texttt{rpm --query --package \textit{package file}}& Display installation status for the local package file\\
  & \texttt{rpm -qi \textit{package}} \newline \texttt{rpm --query --info \textit{package}} &  Display detailed package information \\
  \rowcolor{Gray}
  & \texttt{rpm -qpi \textit{package file}} \newline \texttt{rpm --query --package --info \textit{package file}} &  Display detailed information for the local package file\\
  \textbf{YUM} & \texttt{yum info \textit{package}} &  Display detailed package information, and installation status \\
  \hline
\end{tabular}

\section{Display package dependencies}
\begin{tabular}{ p{3.5cm} p{9cm} p{11cm}}
  \hline
  \rowcolor{Gray}
  \textbf{Tool} & \textbf{Command} & \textbf{Notes} \\
  \hline 
  \textbf{RPM} & \texttt{rpm -qR \textit{package}} \newline \texttt{rpm --query --requires \textit{package}} & Display required packages for an installed package \\
  \rowcolor{Gray}
  & \texttt{rpm -qpR \textit{package file}} \newline \texttt{rpm --query --package --requires \textit{package}} & Display required packages for a package available as a local file\\
  \textbf{YUM} & \texttt{yum deplist \textit{package}} \newline \texttt{repoquery --requires \textit{package}} \newline \texttt{yumdownloader --resolve \textit{package}} & Display dependencies for a package \\
  \hline
\end{tabular}

\newpage

\cheatsheet

\section{List files of a package prior to installation}
\begin{tabular}{ p{3.5cm} p{9cm} p{11cm}}
  \hline
  \rowcolor{Gray}
  \textbf{Tool} & \textbf{Command} & \textbf{Notes} \\
  \hline 
  \textbf{RPM} & \texttt{rpm -ql \textit{package}} \newline \texttt{rpm --query --list \textit{package}} & Show content of local package\\
  \rowcolor{Gray}
  & \texttt{rpm -qpl \textit{package file}} \newline \texttt{rpm --query --package --list \textit{package file}} & Show content of local, or remote package\\
  \textbf{YUM} & \texttt{repoquery -l \textit{package}} \newline \texttt{repoquery --list \textit{package}} & Show content of local, or remote package\\
  \hline
\end{tabular}

% -- TBD 

\section{List files of an installed package}
\begin{tabular}{ p{3.5cm} p{9cm} p{11cm}}
  \hline
  \rowcolor{Gray}
  \textbf{Tool} & \textbf{Command} & \textbf{Notes} \\
  \hline 
  \textbf{RPM} & \texttt{rpm -ql \textit{package}} \newline \texttt{rpm --query --list \textit{package}} & Show content of local package\\
  \rowcolor{Gray}
  \textbf{YUM} & \texttt{repoquery -l \textit{package}} \newline \texttt{repoquery --list \textit{package}} & Show content of local, or remote package\\
  \hline
\end{tabular}

\section{List configuration files of a package}
\begin{tabular}{ p{3.5cm} p{9cm} p{11cm}}
  \hline
  \rowcolor{Gray}
  \textbf{Tool} & \textbf{Command} & \textbf{Notes} \\
  \hline 
  \textbf{RPM} & \texttt{rpm -qc \textit{package}} \newline \texttt{rpm --query --configfiles \textit{package}} & List all the files from the package that are explicitly flagged as confi\-guration files\\
  \hline
\end{tabular}

\section{List documentation files of a package}
\begin{tabular}{ p{3.5cm} p{9cm} p{11cm}}
  \hline
  \rowcolor{Gray}
  \textbf{Tool} & \textbf{Command} & \textbf{Notes} \\
  \hline 
  \textbf{RPM} & \texttt{rpm -qd \textit{package}} \newline \texttt{rpm --query --docfiles \textit{package}} & List all the files from the package that are explicitly flagged as documentation files\\
  \hline
\end{tabular}
\newpage

\cheatsheet

\section{Identify the package from which a file originates}
\begin{tabular}{ p{3.5cm} p{9cm} p{11cm}}
  \hline
  \rowcolor{Gray}
  \textbf{Tool} & \textbf{Command} & \textbf{Notes} \\
  \hline 
  \textbf{RPM} & \texttt{rpm -qf \textit{filename}} \newline \texttt{rpm --query --file \textit{filename}} & Search for the file in all installed packages. Filename has to be specified with full path\\
  \rowcolor{Gray}
  \textbf{YUM} & \texttt{yum provides \textit{filename}} & Search for the file in all available packages. Filename has to be specified with full path\\
  \hline
\end{tabular}

\section{Display package status}
\begin{tabular}{ p{3.5cm} p{9cm} p{11cm}}
  \hline
  \rowcolor{Gray}
  \textbf{Tool} & \textbf{Command} & \textbf{Notes} \\
  \hline 
  \textbf{RPM} & \texttt{rpm -qi \textit{package}} \newline \texttt{rpm --query --info \textit{package}} & Search in all available packages\\
  \rowcolor{Gray}
  & \texttt{rpm -qip \textit{package file}} \newline \texttt{rpm --query --info --package \textit{package file}} & Search in the locally available package file\\
  \textbf{YUM} & \texttt{yum info \textit{package}} & Search in all installed packages\\
  \hline
\end{tabular}

\section{Validate package signature}
\begin{tabular}{ p{3.5cm} p{9cm} p{11cm}}
  \hline
  \rowcolor{Gray}
  \textbf{Tool} & \textbf{Command} & \textbf{Notes} \\
  \hline 
  \textbf{RPM} & \texttt{rpm -K \textit{package}} \newline \texttt{rpm --checksig \textit{package}} & Validate the GPG signature of the package \\
  \hline
\end{tabular}

\section{Check package for changes}
\begin{tabular}{ p{3.5cm} p{9cm} p{11cm}}
  \hline
  \rowcolor{Gray}
  \textbf{Tool} & \textbf{Command} & \textbf{Notes} \\
  \hline 
  \textbf{RPM} & \texttt{rpm -Vp \textit{package file}} \newline \texttt{rpm --verify --package \textit{package file}} & Check the local package file\\
  \rowcolor{Gray}
  & \texttt{rpm -Va \newline \texttt{rpm --verify --all }} & Check all packages \\
  \textbf{YUM} & \texttt{yum verify \textit{package}} & Validate the installed package; requires YUM plugin \textit{yum-verify} \\
  \hline
\end{tabular}

\newpage

\cheatsheet

\section{Display available packages}
\begin{tabular}{ p{3.5cm} p{9cm} p{11cm}}
  \hline
  \rowcolor{Gray}
  \textbf{Tool} & \textbf{Command} & \textbf{Notes} \\
  \hline 
  \textbf{YUM} & \texttt{yum list available} & List all available packages\\
  \hline
\end{tabular}

\section{Display renewable packages}
\begin{tabular}{ p{3.5cm} p{9cm} p{11cm}}
  \hline
  \rowcolor{Gray}
  \textbf{Tool} & \textbf{Command} & \textbf{Notes} \\
  \hline 
  \textbf{YUM} & \texttt{yum check-update} & Query repositories for available package updates\\
  \rowcolor{Gray}
  & \texttt{yum updateinfo list available} & List available packages with a new version\\
  & \texttt{yum updateinfo list sec} \newline \texttt{yum updateinfo list --security} & List available security updates, only \\
  \hline
\end{tabular}

\section{Recommended books and resources}

\begin{description}

    \item[Edward C. Bailey: Maximum RPM] ~ \\
        \begin{tabbing}
            \= Available languages: \= \= English \\
            \> Project website: \> \> \href{https://ftp.osuosl.org/pub/rpm/max-rpm/}{https://ftp.osuosl.org/pub/rpm/max-rpm/} \\
            \> Contribute: \> \> not known \\
        \end{tabbing}
        
    \item[RPM Documentation] ~ \\
        \begin{tabbing}
            \= Available languages: \= \= English \\
            \> Project website: \> \> \href{https://rpm.org/documentation.html}{https://rpm.org/documentation.html} \\
            \> Contribute: \> \> not known \\
        \end{tabbing}
    \end{description}
\end{document}
