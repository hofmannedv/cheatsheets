% Bash cheatsheets
% Version 0.7
% (C) 2023,2024 Frank Hofmann <info@efho.de>
% Available from https://github.com/hofmannedv/cheatsheets

% Published under Creative Commons CC-BY-SA 4.0 International License
% https://creativecommons.org/licenses/by-sa/4.0/

\documentclass[10pt,a4paper]{article}

\usepackage{cheatsheeta4}

\renewcommand{\cheatsheetTitle}{Bash Cheatsheet}
\renewcommand{\cheatsheetVersion}{Version 0.7}
\renewcommand{\cheatsheetCopyright}{
 \copyright~ 2023,2024 Frank Hofmann \faIcon{envelope} \href{mailto:info@efho.de}{info@efho.de} \\ 
 Veröffentlicht unter \href{https://creativecommons.org/licenses/by-sa/4.0/}{Creative Commons CC-BY-SA 4.0 \\ International License}. Erstellt mit \LaTeX. \\ \faIcon{github} \href{https://github.com/hofmannedv/cheatsheets}{https://github.com/hofmannedv/cheatsheets}~.
}

% add document metadata as XMP data
% add keywords based on hyperref package
\usepackage{hyperxmp}
\hypersetup{
    pdfauthor={Frank Hofmann}, 
    pdftitle={Bash Cheatsheet},
    pdfkeywords={Bash, Kommandozeile, Tastenkürzel, Bedienung},
    pdfsubject={Tastenkürzel für den effektiven Umgang mit der Bash},
    pdfcopyright={Copyright (C) 2023,2024 Frank Hofmann. Veröffentlicht unter Creative Commons Attribution - Share Alike 4.0 International License (CC-BY-SA-4.0)}
    pdfcopyrighturl={https://creativecommons.org/licenses/by-sa/4.0/legalcode.en}
}

\begin{document}

\cheatsheet

\begin{multicols}{2}

\section{Bewegen des Cursors}
\begin{tabular}{ p{2.5cm} p{8.5cm} }
  \hline
  \cellSpaceNormal\keyStrg+\key{a} & Gehe zum Anfang der Zeile (auch \key{Pos1})\\
  \rowcolor{Gray}
  \cellSpaceNormal\keyStrg+\key{e} & Gehe zum Ende der Zeile (auch \key{Ende})\\
  \cellSpaceNormal\keyAlt+\key{b} & Gehe ein Wort nach links \\
  \rowcolor{Gray}
  \cellSpaceNormal\keyAlt+\key{f} & Gehe ein Wort nach rechts  \\
  \cellSpaceNormal\keyStrg+\key{b} & Gehe ein Zeichen nach links (auch \key{$\leftarrow$}) \\
  \rowcolor{Gray}
  \cellSpaceNormal\keyStrg+\key{f} & Gehe ein Zeichen nach rechts (auch \key{$\rightarrow$})\\
  \cellSpaceNormal\keyStrg+\key{\textit{xx}} & Schalte zwischen dem Anfang der Zeile und der Cursorposition \textit{xx} um \cellSpaceLittle \\
  \rowcolor{Gray}
  \cellSpaceNormal\key{Tab} & Automatische Vervollständigung für Datei- und \newline Verzeichnisnamen \cellSpaceLittle \\
  \cellSpaceNormal\keyStrg+\key{p} & Gehe zum vorherigen Kommando (auch \key{$\uparrow$}) \cellSpaceLittle \\
  \rowcolor{Gray}
  \cellSpaceNormal\keyStrg+\key{n} & Gehe zum nächsten Kommando (auch \key{$\downarrow$})\\
  \hline
\end{tabular}

%\vfill

\section{Kopieren und einfügen}
\begin{tabular}{ p{2.5cm} p{8.5cm} }
  \hline
  \cellSpaceNormal\keyStrg+\key{w} & Schneide das Wort vor der akt. Cursorposition aus \cellSpaceLittle \\
  \rowcolor{Gray}
  \cellSpaceNormal\keyStrg+\key{k} & Schneide das Wort nach der akt. Cursorposition aus \cellSpaceLittle \\
  \cellSpaceNormal\keyStrg+\key{u} & Schneide den Inhalt der Zeile vor der aktuellen Cursorposition aus \cellSpaceLittle \\
  \rowcolor{Gray}
  \cellSpaceNormal\keyStrg+\key{y} & Füge das zuvor ausgeschnittene Wort oder die Zeile ein \\
  \hline
\end{tabular}

\columnbreak

\section{Löschen von Zeichen und Worten\phantom{g}}
\begin{tabular}{ p{2.5cm} p{8.5cm} }
  \hline
  \cellSpaceNormal\keyStrg+\key{l} & Lösche den Bildschirm \\
  \rowcolor{Gray}
  \cellSpaceNormal\keyAlt+\key{Entf} & Lösche das Wort vor dem Cursor \\
  \cellSpaceNormal\keyAlt+\key{d} & Lösche das Wort nach dem Cursor \\
  \rowcolor{Gray}
  \cellSpaceNormal\keyStrg+\key{d} & Lösche das aktuelle Zeichen (auch \key{Entf})\\
  \cellSpaceNormal\keyStrg+\key{h} & Lösche das Zeichen vor dem Cursor (\keyBackspace)\\
  \rowcolor{Gray}
  \cellSpaceNormal\keyStrg+\key{k} & Lösche alle Zeichen bis zum Ende der Zeile \cellSpaceLittle \\
  \cellSpaceNormal\keyStrg+\key{x} + \newline \cellSpaceLittle\key{Backspace} & Lösche alle Zeichen bis zum Anfang der Zeile \cellSpaceLittle \\
  \hline
\end{tabular}

%\vfill

\section{Schreibweise ändern}
\begin{tabular}{ p{2.5cm} p{8.5cm} }
  \hline 
  \cellSpaceNormal\keyAlt+\key{l} & Wandele jedes Zeichen in einen Kleinbuchstaben um (Cursorposition bis zum Ende des aktuellen Wortes) \cellSpaceLittle \\
  \rowcolor{Gray}
  \cellSpaceNormal\keyAlt+\key{u} & Wandele jedes Zeichen in einen Groß\-buchstaben um (Cursorposition bis zum Ende des aktuellen Wortes) \cellSpaceLittle \\
  \cellSpaceNormal\keyAlt+\key{c} & Wandele den Buchstaben an der Cursorposition in einen Großbuchstaben um und springe zum Ende des Wortes \cellSpaceLittle \\
  \rowcolor{Gray}
  \cellSpaceNormal\keyAlt+\key{r} & Änderungen abbrechen und Originalzeile wiederherstellen (revert) \cellSpaceLittle \\
  \cellSpaceNormal\keyAlt+\key{\_} & Letzte Aktion rückgängig machen \\
  \hline
\end{tabular}

\end{multicols}

\newpage

\cheatsheet

\begin{multicols}{2}

\section{Zeichen und Worte vertauschen\phantom{g}}
\begin{tabular}{ p{2.5cm} p{8.5cm} }
  \hline
  \cellSpaceNormal\keyAlt+\key{t} & Tausche das aktuelle Wort mit dem vorhergehenden aus \cellSpaceLittle\\
  \rowcolor{Gray}
  \cellSpaceNormal\keyStrg+\key{t} & Vertausche die letzten beiden Zeichen vor dem Cursor \\
  \cellSpaceNormal\key{Esc}+\key{t} & Vertausche die letzten beiden Worte vor dem Cursor \cellSpaceLittle\\
  \hline
\end{tabular}

\vfill

\section{Kommandozeilenhistorie}

\begin{tabular}{ p{2.5cm} p{8.5cm} }
  \hline
  \cellSpaceNormal \key{Strg}+\key{r} & Suche rückwärts in der Kommandozeilenhistorie. Rufe das letzte Kommando auf, welches die angegebenen Zeichen beinhaltet \cellSpaceLittle \\
  \rowcolor{Gray}
  \cellSpaceNormal \keyStrg+\key{p} & Gehe zum vorhergehenden Kommando in der Historie \\
  \cellSpaceNormal \keyStrg+\key{n} & Gehe zum nächstes Kommando in der Historie \cellSpaceLittle \\
  \rowcolor{Gray}
  \cellSpaceNormal \keyStrg+\key{s} &  Gehe zurück zum nächsten, passendsten Kommando \cellSpaceLittle \\
  \cellSpaceNormal \keyStrg+\key{o} & Führe das Kommando aus, welches vorher via \newline \keyStrg+\key{r} oder \keyStrg+\key{s} gefunden wurde\cellSpaceNormal\\
  \rowcolor{Gray}
  \cellSpaceNormal \keyStrg+\key{g} & Verlasse den Suchmodus der Historie \\
  \cellSpaceNormal \key{!}~\key{!} & Wiederhole das vorhergehende Kommando\cellSpaceLittle \\
  \rowcolor{Gray}
  \cellSpaceNormal \key{Esc}+\key{.}, \newline \cellSpaceNormal \key{!\$} oder \newline \cellSpaceNormal \keyAlt+\key{.}\cellSpaceLittle & Hole das letzte Argument des vorhergehenden Kommandos \\
  \hline
\end{tabular}

\columnbreak

\section{Prozesssteuerung}
\begin{tabular}{ p{2.5cm} p{8.5cm} }
  \hline
  \cellSpaceNormal \key{Strg}+\key{c} & Unterbreche den Prozess, der ausgeführt wird (SIGINT) \\
  \rowcolor{Gray}
  \cellSpaceNormal \keyStrg+\key{l} & Lösche den Bildschirm \\
  \cellSpaceNormal \keyStrg+\key{s} & Beende die Ausgabe auf dem Bildschirm (für langlaufende, ausführliche Kommandos), benutze danach die Bewegungstasten \key{Bild $\uparrow$} und \key{Bild $\downarrow$}, um in der Ausgabe zu blättern \cellSpaceNormal\\
  \rowcolor{Gray}
  \cellSpaceNormal \keyStrg+\key{q} & Erlaube die Ausgabe auf dem Bildschirm (falls diese zuvor mit dem vorherigen Kommando unterbrochen wurde) \cellSpaceLittle \\
  \cellSpaceNormal \keyStrg+\key{d} & Sende ein EOF-Zeichen sofern nicht durch eine Option abgeschaltet. Das schließt die aktuelle Shell (EXIT) \cellSpaceLittle\\
  \rowcolor{Gray}
  \cellSpaceNormal \keyStrg+\key{z} & Sende das Signal SIGTSTP an den aktuellen Prozess, dass ihn unterbricht. Um zu diesem später zurückzukehren, rufe \texttt{fg \textit{Auftragsummer}} auf (hole den Prozess wieder in den Vordergrund) \cellSpaceLittle \\
  \hline
\end{tabular}

%\vfill

\section{Tips und Tricks}
%\begin{tabular}{ p{11.4cm}}
\begin{tabular}{ p{4.5cm} p{6.5cm} }
  \hline
~ & ~ \\
\multicolumn{2}{p{11.2cm}}{Sofern nicht anderweitig konfiguriert, läuft die Bash im Emacs-Modus. Sie schalten zwischen dem Vi- und Emacs-Modus wie folgt um:} \\
~ & ~ \\
Wechsel zum Vi-Modus: & \texttt{set -o vi} \\ 
Wechsel zum Emacs-Modus: & \texttt{set -o emacs} \\ 
~ & ~ \\
Anzeige der Tastenkürzel: & \texttt{bind -p} \\ 
~ & ~ \\
  \hline
\end{tabular}
%\vfill 

\end{multicols}
\end{document}
